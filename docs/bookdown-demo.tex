\documentclass[spanish,]{book}
\usepackage{lmodern}
\usepackage{amssymb,amsmath}
\usepackage{ifxetex,ifluatex}
\usepackage{fixltx2e} % provides \textsubscript
\ifnum 0\ifxetex 1\fi\ifluatex 1\fi=0 % if pdftex
  \usepackage[T1]{fontenc}
  \usepackage[utf8]{inputenc}
\else % if luatex or xelatex
  \ifxetex
    \usepackage{mathspec}
  \else
    \usepackage{fontspec}
  \fi
  \defaultfontfeatures{Ligatures=TeX,Scale=MatchLowercase}
\fi
% use upquote if available, for straight quotes in verbatim environments
\IfFileExists{upquote.sty}{\usepackage{upquote}}{}
% use microtype if available
\IfFileExists{microtype.sty}{%
\usepackage{microtype}
\UseMicrotypeSet[protrusion]{basicmath} % disable protrusion for tt fonts
}{}
\usepackage{hyperref}
\hypersetup{unicode=true,
            pdftitle={Introducción al Análisis de Datos},
            pdfauthor={Matías Alfonso},
            pdfborder={0 0 0},
            breaklinks=true}
\urlstyle{same}  % don't use monospace font for urls
\ifnum 0\ifxetex 1\fi\ifluatex 1\fi=0 % if pdftex
  \usepackage[shorthands=off,main=spanish]{babel}
\else
  \usepackage{polyglossia}
  \setmainlanguage[]{spanish}
\fi
\usepackage{natbib}
\bibliographystyle{apalike}
\usepackage{color}
\usepackage{fancyvrb}
\newcommand{\VerbBar}{|}
\newcommand{\VERB}{\Verb[commandchars=\\\{\}]}
\DefineVerbatimEnvironment{Highlighting}{Verbatim}{commandchars=\\\{\}}
% Add ',fontsize=\small' for more characters per line
\usepackage{framed}
\definecolor{shadecolor}{RGB}{248,248,248}
\newenvironment{Shaded}{\begin{snugshade}}{\end{snugshade}}
\newcommand{\KeywordTok}[1]{\textcolor[rgb]{0.13,0.29,0.53}{\textbf{#1}}}
\newcommand{\DataTypeTok}[1]{\textcolor[rgb]{0.13,0.29,0.53}{#1}}
\newcommand{\DecValTok}[1]{\textcolor[rgb]{0.00,0.00,0.81}{#1}}
\newcommand{\BaseNTok}[1]{\textcolor[rgb]{0.00,0.00,0.81}{#1}}
\newcommand{\FloatTok}[1]{\textcolor[rgb]{0.00,0.00,0.81}{#1}}
\newcommand{\ConstantTok}[1]{\textcolor[rgb]{0.00,0.00,0.00}{#1}}
\newcommand{\CharTok}[1]{\textcolor[rgb]{0.31,0.60,0.02}{#1}}
\newcommand{\SpecialCharTok}[1]{\textcolor[rgb]{0.00,0.00,0.00}{#1}}
\newcommand{\StringTok}[1]{\textcolor[rgb]{0.31,0.60,0.02}{#1}}
\newcommand{\VerbatimStringTok}[1]{\textcolor[rgb]{0.31,0.60,0.02}{#1}}
\newcommand{\SpecialStringTok}[1]{\textcolor[rgb]{0.31,0.60,0.02}{#1}}
\newcommand{\ImportTok}[1]{#1}
\newcommand{\CommentTok}[1]{\textcolor[rgb]{0.56,0.35,0.01}{\textit{#1}}}
\newcommand{\DocumentationTok}[1]{\textcolor[rgb]{0.56,0.35,0.01}{\textbf{\textit{#1}}}}
\newcommand{\AnnotationTok}[1]{\textcolor[rgb]{0.56,0.35,0.01}{\textbf{\textit{#1}}}}
\newcommand{\CommentVarTok}[1]{\textcolor[rgb]{0.56,0.35,0.01}{\textbf{\textit{#1}}}}
\newcommand{\OtherTok}[1]{\textcolor[rgb]{0.56,0.35,0.01}{#1}}
\newcommand{\FunctionTok}[1]{\textcolor[rgb]{0.00,0.00,0.00}{#1}}
\newcommand{\VariableTok}[1]{\textcolor[rgb]{0.00,0.00,0.00}{#1}}
\newcommand{\ControlFlowTok}[1]{\textcolor[rgb]{0.13,0.29,0.53}{\textbf{#1}}}
\newcommand{\OperatorTok}[1]{\textcolor[rgb]{0.81,0.36,0.00}{\textbf{#1}}}
\newcommand{\BuiltInTok}[1]{#1}
\newcommand{\ExtensionTok}[1]{#1}
\newcommand{\PreprocessorTok}[1]{\textcolor[rgb]{0.56,0.35,0.01}{\textit{#1}}}
\newcommand{\AttributeTok}[1]{\textcolor[rgb]{0.77,0.63,0.00}{#1}}
\newcommand{\RegionMarkerTok}[1]{#1}
\newcommand{\InformationTok}[1]{\textcolor[rgb]{0.56,0.35,0.01}{\textbf{\textit{#1}}}}
\newcommand{\WarningTok}[1]{\textcolor[rgb]{0.56,0.35,0.01}{\textbf{\textit{#1}}}}
\newcommand{\AlertTok}[1]{\textcolor[rgb]{0.94,0.16,0.16}{#1}}
\newcommand{\ErrorTok}[1]{\textcolor[rgb]{0.64,0.00,0.00}{\textbf{#1}}}
\newcommand{\NormalTok}[1]{#1}
\usepackage{longtable,booktabs}
\usepackage{graphicx,grffile}
\makeatletter
\def\maxwidth{\ifdim\Gin@nat@width>\linewidth\linewidth\else\Gin@nat@width\fi}
\def\maxheight{\ifdim\Gin@nat@height>\textheight\textheight\else\Gin@nat@height\fi}
\makeatother
% Scale images if necessary, so that they will not overflow the page
% margins by default, and it is still possible to overwrite the defaults
% using explicit options in \includegraphics[width, height, ...]{}
\setkeys{Gin}{width=\maxwidth,height=\maxheight,keepaspectratio}
\IfFileExists{parskip.sty}{%
\usepackage{parskip}
}{% else
\setlength{\parindent}{0pt}
\setlength{\parskip}{6pt plus 2pt minus 1pt}
}
\setlength{\emergencystretch}{3em}  % prevent overfull lines
\providecommand{\tightlist}{%
  \setlength{\itemsep}{0pt}\setlength{\parskip}{0pt}}
\setcounter{secnumdepth}{5}
% Redefines (sub)paragraphs to behave more like sections
\ifx\paragraph\undefined\else
\let\oldparagraph\paragraph
\renewcommand{\paragraph}[1]{\oldparagraph{#1}\mbox{}}
\fi
\ifx\subparagraph\undefined\else
\let\oldsubparagraph\subparagraph
\renewcommand{\subparagraph}[1]{\oldsubparagraph{#1}\mbox{}}
\fi

%%% Use protect on footnotes to avoid problems with footnotes in titles
\let\rmarkdownfootnote\footnote%
\def\footnote{\protect\rmarkdownfootnote}

%%% Change title format to be more compact
\usepackage{titling}

% Create subtitle command for use in maketitle
\providecommand{\subtitle}[1]{
  \posttitle{
    \begin{center}\large#1\end{center}
    }
}

\setlength{\droptitle}{-2em}

  \title{Introducción al Análisis de Datos}
    \pretitle{\vspace{\droptitle}\centering\huge}
  \posttitle{\par}
    \author{Matías Alfonso}
    \preauthor{\centering\large\emph}
  \postauthor{\par}
      \predate{\centering\large\emph}
  \postdate{\par}
    \date{2019-08-07}

\usepackage{booktabs}
\usepackage{amsthm}
\makeatletter
\def\thm@space@setup{%
  \thm@preskip=8pt plus 2pt minus 4pt
  \thm@postskip=\thm@preskip
}
\makeatother

\begin{document}
\maketitle

{
\setcounter{tocdepth}{1}
\tableofcontents
}
\part{Programación en R}\label{part-programacion-en-r}

\chapter{Preliminares}\label{prelim}

R es un lenguaje de programación desarrollado inicialmente por Ross
Ihaka y Robert Gentleman en el departamento de Estadística de la
Universidad de Auckland en 1993. Está orientado específicamente con un
enfoque al análisis estadístico.\\
R se desarrolla a partir de un lenguaje denominado S, desarrollado por
John Chambers en 1976, disponible a partir del software comercial
S-PLUS.\\
Es un lenguaje interactivo, permite la ejecución de instrucciónes en
líneas de comando en una consola.

\section{¿Por qué R?}\label{por-que-r}

R puede ser ejecutado en múltiples plataformas y en la gran mayoría de
los sistemas operativos. Puede ser ejecutado en tablets, teléfonos o
computadoras. La utilización de scripts permite compartir fácilmente los
análisis con los colegas, así como asegurar la reproductibilidad de los
resultados. Todo lo que realizamos mediante una interfaz gráfica con el
mouse no deja registros de nuestro trabajo e impide que podamos repasar
nuestro trabajo para corregir errores.\\
La versatilidad y la potencia que otorga un lenguaje de programación es
mucho mayor que la que podemos obtener con softwares estadísticos de
interfaz gráfica.\\
La comunidad de usuarios y desarrolladores de R está en constante
crecimiento en los últimos años. Hay una enorme cantidad de gente
realizando nuevos desarrollos en R cada día, que están a la vanguardia
de la ciencia computacional y estadística.

\section{Software Libre}\label{software-libre}

La mayor ventaja que tiene R con respecto a otros softwares de análisis
estadístico es que es un software libre. ¿Qué quiere decir eso? Por un
lado, que es gratuito. Por otro, que el código fuente con el que R fue
desarrollado está abierto, se puede descargar y está diponible online.
Actualmente el copyrigth de R lo posee la
\href{https://www.r-project.org/foundation/}{R Foundation}. R forma
parte del \href{https://es.wikipedia.org/wiki/GNU}{sistema GNU},
desarrollado por la {[}Free Software Foundation{]}(. De acuerdo a la
\href{https://es.wikipedia.org/wiki/Free_Software_Foundation}{Free
Software Foundation}, con el software libre se garantizan cuatro
libertades fundamentales:

\begin{itemize}
\tightlist
\item
  La libertad de ejecutar el programa para cualquier propósito.
  (Libertad 0)
\item
  La libertad de estudiar cómo el programa funciona y adaptarlo a tus
  propias necesidades. (Libertad 1)
\item
  La libertad de redistribuir copias de manera que puedas ayudar a
  alguien. (Libertad 2)
\item
  La libertad de mejorar el programa, y liberar tus mejoras al público,
  de manera que se beneficie toda la comunidad. (Libertad 3)
\end{itemize}

\section{Sistema de Paquetes}\label{sistema-de-paquetes}

El sistema de funcionalidades de R se encuentra agrupado en paquetes. La
mayor parte de los paquetes se encuentran disponibles en
\href{https://cran.r-project.org/}{Comprehensive R Archive Network
(CRAN)}. Hay un conjunto de paquetes principales, de base, que incluye
todos los paquetes que se instalan por defecto cuando instalamos R.
Luego, tenemos un montón de paquetes con funcionalidades específicas que
podemos instalar en función de nuestras necesidades.

\chapter{Primeros pasos}\label{primeros-pasos}

\section{Asignación de datos y
evaluación.}\label{asignacion-de-datos-y-evaluacion.}

R es un \emph{lenguaje interpretado}. Esto quiere decir que le podemos
ir pasando instrucciones y el programama las irá interpretando. Cuando
ejecutamos el programa, nos encontramos con el prompt a la espera de
intrucciones:

\begin{verbatim}
>
\end{verbatim}

Una de las operaciones más sencillas que podemos realizar es la
asignación de valores a las variables. El operador de asignación es
\texttt{\textless{}-}.

\begin{Shaded}
\begin{Highlighting}[]
\NormalTok{x <-}\StringTok{ }\DecValTok{1}
\KeywordTok{print}\NormalTok{(x)}
\end{Highlighting}
\end{Shaded}

\begin{verbatim}
## [1] 1
\end{verbatim}

\begin{Shaded}
\begin{Highlighting}[]
\NormalTok{x}
\end{Highlighting}
\end{Shaded}

\begin{verbatim}
## [1] 1
\end{verbatim}

\begin{Shaded}
\begin{Highlighting}[]
\NormalTok{texto <-}\StringTok{ "hola mundo"}
\NormalTok{texto}
\end{Highlighting}
\end{Shaded}

\begin{verbatim}
## [1] "hola mundo"
\end{verbatim}

Podemos imprimir el valor de una variable con la función
\texttt{print()} o directamente escribiendo la variable.

Tenemos dos formas de interactuar con R:

\begin{itemize}
\tightlist
\item
  Tipear directamente los comandos en el prompt y ejecutarlos.
\item
  Escribir un archivo de texto con todas las intrucciones y luego
  ejecutarlo. Este archivo se denomina script.
\end{itemize}

\section{Working directory}\label{working-directory}

Lo primero que debemos hacer cuando comenzamos a trabajar en R es
configurar el directorio de trabajo. Una buena costumbre es crear un
directorio nuevo de trabajo cuando comenzamos un proyecto nuevo. Luego
configuramos esa carpeta como directorio de trabajo. Colocamos allí
todos los archivos vinculados a ese proyecto. Para determinar en qué
directorio estamos parados, podemos utilizar el comando
\texttt{getwd()}. Para configurar el directorio de trabajo, utilizamos

\begin{verbatim}
setwd(#RUTA-A-DIRECTORIO)
\end{verbatim}

\section{Comentarios}\label{comentarios}

Todo lo que escribamos luego de un \texttt{\#} en una intrucción, no
será evaluado.

\begin{Shaded}
\begin{Highlighting}[]
\NormalTok{x <-}\StringTok{ }\KeywordTok{c}\NormalTok{(}\DecValTok{3}\NormalTok{, }\DecValTok{4}\NormalTok{, }\DecValTok{5}\NormalTok{)}
\NormalTok{## Esto no se ejecuta}

\NormalTok{x}
\end{Highlighting}
\end{Shaded}

\begin{verbatim}
## [1] 3 4 5
\end{verbatim}

Ello nos permite comentar el código que escribimos, a manera de
documentación.

\section{Objetos básicos en R}\label{objetos-basicos-en-r}

Casi todo lo que encontremos en R, se denominan \emph{objetos}. Hay 5
tipos de objetos básicos o atómicos:

\begin{itemize}
\tightlist
\item
  lógico
\item
  numérico
\item
  entero
\item
  complejo
\item
  caracter
\end{itemize}

Veamos algunos ejemplos:

\begin{Shaded}
\begin{Highlighting}[]
\NormalTok{## Logico}
\OtherTok{TRUE}
\end{Highlighting}
\end{Shaded}

\begin{verbatim}
## [1] TRUE
\end{verbatim}

\begin{Shaded}
\begin{Highlighting}[]
\OtherTok{FALSE}
\end{Highlighting}
\end{Shaded}

\begin{verbatim}
## [1] FALSE
\end{verbatim}

\begin{Shaded}
\begin{Highlighting}[]
\NormalTok{## Numérico}
\KeywordTok{c}\NormalTok{(}\FloatTok{1.509}\NormalTok{, }\FloatTok{2.859}\NormalTok{)}
\end{Highlighting}
\end{Shaded}

\begin{verbatim}
## [1] 1.509 2.859
\end{verbatim}

\begin{Shaded}
\begin{Highlighting}[]
\NormalTok{## Enteros}
\DecValTok{1}\OperatorTok{:}\DecValTok{10}
\end{Highlighting}
\end{Shaded}

\begin{verbatim}
##  [1]  1  2  3  4  5  6  7  8  9 10
\end{verbatim}

\begin{Shaded}
\begin{Highlighting}[]
\NormalTok{## Caracter}
\StringTok{"casa"}
\end{Highlighting}
\end{Shaded}

\begin{verbatim}
## [1] "casa"
\end{verbatim}

Existen muchos más clases de objetos en R. Para averiguar de que tipo es
un objeto, podemos utilizar la función \texttt{class()}

\begin{Shaded}
\begin{Highlighting}[]
\NormalTok{x <-}\StringTok{ }\DecValTok{1}\OperatorTok{:}\DecValTok{10}
\KeywordTok{class}\NormalTok{(x)}
\end{Highlighting}
\end{Shaded}

\begin{verbatim}
## [1] "integer"
\end{verbatim}

\begin{Shaded}
\begin{Highlighting}[]
\KeywordTok{class}\NormalTok{(}\StringTok{"TRUE"}\NormalTok{)}
\end{Highlighting}
\end{Shaded}

\begin{verbatim}
## [1] "character"
\end{verbatim}

Para preguntar por la clase de un objeto, podemos utilizar el comando
\texttt{class()}

\begin{Shaded}
\begin{Highlighting}[]
\NormalTok{x <-}\StringTok{ }\DecValTok{1}\OperatorTok{:}\DecValTok{10}
\KeywordTok{class}\NormalTok{(x)}
\end{Highlighting}
\end{Shaded}

\begin{verbatim}
## [1] "integer"
\end{verbatim}

\begin{Shaded}
\begin{Highlighting}[]
\NormalTok{y <-}\StringTok{ "casa"}
\KeywordTok{class}\NormalTok{(y)}
\end{Highlighting}
\end{Shaded}

\begin{verbatim}
## [1] "character"
\end{verbatim}

\section{Factores}\label{factores}

Los factores son básicamente objetos de clase entero, pero con
etiquetas. Son datos categóricos y pueden estar ordenados o no.

\begin{Shaded}
\begin{Highlighting}[]
\NormalTok{f <-}\StringTok{ }\KeywordTok{factor}\NormalTok{(}\KeywordTok{c}\NormalTok{(}\StringTok{"si"}\NormalTok{, }\StringTok{"si"}\NormalTok{, }\StringTok{"no"}\NormalTok{, }\StringTok{"si"}\NormalTok{))}
\NormalTok{f}
\end{Highlighting}
\end{Shaded}

\begin{verbatim}
## [1] si si no si
## Levels: no si
\end{verbatim}

\begin{Shaded}
\begin{Highlighting}[]
\NormalTok{f <-}\StringTok{ }\KeywordTok{factor}\NormalTok{(}\KeywordTok{c}\NormalTok{(}\StringTok{"bajo"}\NormalTok{, }\StringTok{"bajo"}\NormalTok{, }\StringTok{"medio"}\NormalTok{, }\StringTok{"alto"}\NormalTok{),}
            \DataTypeTok{levels =} \KeywordTok{c}\NormalTok{(}\StringTok{"bajo"}\NormalTok{, }\StringTok{"medio"}\NormalTok{, }\StringTok{"alto"}\NormalTok{),}
            \DataTypeTok{ordered =} \OtherTok{TRUE}\NormalTok{)}
\NormalTok{f}
\end{Highlighting}
\end{Shaded}

\begin{verbatim}
## [1] bajo  bajo  medio alto 
## Levels: bajo < medio < alto
\end{verbatim}

\section{Cómo buscar ayuda}\label{como-buscar-ayuda}

R tiene un sistema de ayuda integrado. Si queremos saber para qué sirve
un comando determinado o como pasarle los argumentos, podemos utilizar
\texttt{?} o \texttt{help()}. Supongamos que queremos saber cómo se
utiliza la función \texttt{sum()}

\begin{Shaded}
\begin{Highlighting}[]
\KeywordTok{help}\NormalTok{(sum)}

\NormalTok{?vector}
\end{Highlighting}
\end{Shaded}

\chapter{Estructuras de datos}\label{estructuras-de-datos}

\subsection{Vectores}\label{vectores}

La forma más elemental de almacenar datos en R es en un vector. Un
\textbf{vector} es una concatenación de objetos del mismo tipo. Podemos
utilizar la función \texttt{c()} para crear vectores.

\begin{Shaded}
\begin{Highlighting}[]
\NormalTok{x <-}\StringTok{ }\KeywordTok{c}\NormalTok{(}\DecValTok{1}\NormalTok{, }\DecValTok{2}\NormalTok{, }\DecValTok{3}\NormalTok{, }\FloatTok{2.5}\NormalTok{)}
\NormalTok{x <-}\StringTok{ }\KeywordTok{c}\NormalTok{(}\OtherTok{TRUE}\NormalTok{, }\OtherTok{FALSE}\NormalTok{)}
\NormalTok{x <-}\StringTok{ }\KeywordTok{c}\NormalTok{(T, F)}
\NormalTok{x <-}\StringTok{ }\KeywordTok{c}\NormalTok{(}\StringTok{"casa"}\NormalTok{, }\StringTok{"árbol"}\NormalTok{, }\StringTok{"patio"}\NormalTok{)}

\NormalTok{## También podemos utilizar la función vector.}
\NormalTok{x <-}\StringTok{ }\KeywordTok{vector}\NormalTok{(}\DataTypeTok{mode =} \StringTok{"numeric"}\NormalTok{, }\DataTypeTok{length =} \DecValTok{10}\NormalTok{)}
\end{Highlighting}
\end{Shaded}

Si concatenamos elementos de diferente clase, R realizará una coerción
automática de la clase de los objetos.

\begin{Shaded}
\begin{Highlighting}[]
\NormalTok{x <-}\StringTok{ }\KeywordTok{c}\NormalTok{(}\StringTok{"casa"}\NormalTok{, }\DecValTok{2}\NormalTok{) ## character}
\NormalTok{x <-}\StringTok{ }\KeywordTok{c}\NormalTok{(}\OtherTok{TRUE}\NormalTok{, }\DecValTok{2}\NormalTok{) ## numeric}

\KeywordTok{class}\NormalTok{(x)}
\end{Highlighting}
\end{Shaded}

\begin{verbatim}
## [1] "numeric"
\end{verbatim}

\section{Listas}\label{listas}

Las listas también son una concatenación de elementos, pero pueden
contener elementos de diferente clase. Para crear una lista, podemos
utilizar \texttt{list()}

\begin{Shaded}
\begin{Highlighting}[]
\NormalTok{x <-}\StringTok{ }\KeywordTok{list}\NormalTok{(}\StringTok{"peso"}\NormalTok{, }\DecValTok{2}\NormalTok{, }\StringTok{"altura"}\NormalTok{, }\DecValTok{3}\NormalTok{, }\OtherTok{TRUE}\NormalTok{)}
\NormalTok{x}
\end{Highlighting}
\end{Shaded}

\begin{verbatim}
## [[1]]
## [1] "peso"
## 
## [[2]]
## [1] 2
## 
## [[3]]
## [1] "altura"
## 
## [[4]]
## [1] 3
## 
## [[5]]
## [1] TRUE
\end{verbatim}

\section{Matrices}\label{matrices}

Las matrices son vectores, pero con un atributo de dimensión. La
dimensión en sí es un vector de enteros de largo 2.

\begin{Shaded}
\begin{Highlighting}[]
\NormalTok{m <-}\StringTok{ }\KeywordTok{matrix}\NormalTok{(}\DecValTok{1}\OperatorTok{:}\DecValTok{9}\NormalTok{, }\DataTypeTok{nrow =} \DecValTok{3}\NormalTok{)}
\NormalTok{m}
\end{Highlighting}
\end{Shaded}

\begin{verbatim}
##      [,1] [,2] [,3]
## [1,]    1    4    7
## [2,]    2    5    8
## [3,]    3    6    9
\end{verbatim}

\begin{Shaded}
\begin{Highlighting}[]
\KeywordTok{dim}\NormalTok{(m)}
\end{Highlighting}
\end{Shaded}

\begin{verbatim}
## [1] 3 3
\end{verbatim}

Al igual que los vectores, contienen objetos de la misma clase. Las
matrices tienen algunas propiedades matemáticas interesantes, pues se
pueden realizar operaciones especiales con ellas, por ejemplo, se pueden
sumar o multiplicar.

\section{Data frames}\label{data-frames}

Los data frames son datos tabulados. Son tablas, donde cada columna
puede ser de una clase diferente. Es un objeto particularmente útil para
el análisis estadístico.

\begin{Shaded}
\begin{Highlighting}[]
\KeywordTok{data.frame}\NormalTok{(}\DataTypeTok{Id =} \KeywordTok{c}\NormalTok{(}\DecValTok{1}\NormalTok{, }\DecValTok{2}\NormalTok{, }\DecValTok{3}\NormalTok{),}
           \DataTypeTok{Nombre =} \KeywordTok{c}\NormalTok{(}\StringTok{"Juan"}\NormalTok{, }\StringTok{"Carlos"}\NormalTok{, }\StringTok{"Ramona"}\NormalTok{),}
           \DataTypeTok{Altura =} \KeywordTok{c}\NormalTok{(}\FloatTok{1.76}\NormalTok{, }\FloatTok{1.80}\NormalTok{, }\FloatTok{1.65}\NormalTok{))}
\end{Highlighting}
\end{Shaded}

\begin{verbatim}
##   Id Nombre Altura
## 1  1   Juan   1.76
## 2  2 Carlos   1.80
## 3  3 Ramona   1.65
\end{verbatim}

\section{Valores faltantes}\label{valores-faltantes}

Existen dos tipos de valores faltantes en R:

\begin{itemize}
\tightlist
\item
  \texttt{NA}
\item
  \texttt{NaN}
\end{itemize}

\chapter{Obteniendo datos}\label{obteniendo-datos}

Existen una enorme cantidad de funciones para abrir archivos de diversos
tipos.

\section{Datos tabulares}\label{datos-tabulares}

Un formato estándar y abierto para guardar información en forma de
tablas son archivos separados por comas (.csv) Para leer estos datos
podemos utilizar:

\begin{itemize}
\tightlist
\item
  \texttt{read.tables()}
\item
  \texttt{read.csv()}
\end{itemize}

\begin{Shaded}
\begin{Highlighting}[]
\NormalTok{## Leemos los datos desde un archivo y los guardamos en la variable base}
\NormalTok{base <-}\StringTok{ }\KeywordTok{read.csv}\NormalTok{(}\StringTok{"data/titanic.csv"}\NormalTok{)}

\NormalTok{## Imprimimos las primeras 5 filas de la base}
\KeywordTok{head}\NormalTok{(base)[, }\DecValTok{1}\OperatorTok{:}\DecValTok{4}\NormalTok{]}
\end{Highlighting}
\end{Shaded}

\begin{verbatim}
##   PassengerId Survived Pclass
## 1           1        0      3
## 2           2        1      1
## 3           3        1      3
## 4           4        1      1
## 5           5        0      3
## 6           6        0      3
##                                                  Name
## 1                             Braund, Mr. Owen Harris
## 2 Cumings, Mrs. John Bradley (Florence Briggs Thayer)
## 3                              Heikkinen, Miss. Laina
## 4        Futrelle, Mrs. Jacques Heath (Lily May Peel)
## 5                            Allen, Mr. William Henry
## 6                                    Moran, Mr. James
\end{verbatim}

También podemos leer datos directamente de la web

\begin{Shaded}
\begin{Highlighting}[]
\NormalTok{## Datos Abiertos}
\NormalTok{## Consulta de Medicamentos esenciales}
\NormalTok{salud <-}\StringTok{ }\KeywordTok{read.csv}\NormalTok{(}\StringTok{"http://datos.salud.gob.ar/dataset/5fcacd04-58eb-4b43-89a0-55231c58f1b4/resource/ed9e418f-9858-44df-8ce7-a74fde738684/download/consultas-medicamentos-esenciales.csv"}\NormalTok{)}

\KeywordTok{head}\NormalTok{(salud)}
\end{Highlighting}
\end{Shaded}

\begin{verbatim}
##   provincia_id                  provincia_desc      año consultas_cantidad
## 1            2 Ciudad Autonoma de Buenos Aires año_2003             559785
## 2            6                    Buenos Aires año_2003            9544395
## 3           10                       Catamarca año_2003             372199
## 4           14                         Cordoba año_2003            3491961
## 5           18                      Corrientes año_2003             764887
## 6           22                           Chaco año_2003            1476825
\end{verbatim}

\part{Introducción al análisis
estadístico}\label{part-introduccion-al-analisis-estadistico}

\chapter{Operaciones básicas}\label{operaciones-basicas}

Comencemos calculando una proporción. Supongamos que realizamos 10
mediciones para la variables rendimiento escolar. La variable puede
tomar tres valores: bajo, medio y alto.

\begin{Shaded}
\begin{Highlighting}[]
\NormalTok{rendimiento <-}\StringTok{ }\KeywordTok{c}\NormalTok{(}\DecValTok{1}\NormalTok{, }\DecValTok{3}\NormalTok{, }\DecValTok{1}\NormalTok{, }\DecValTok{2}\NormalTok{, }\DecValTok{2}\NormalTok{, }\DecValTok{1}\NormalTok{, }\DecValTok{2}\NormalTok{, }\DecValTok{2}\NormalTok{, }\DecValTok{3}\NormalTok{, }\DecValTok{2}\NormalTok{)}

\NormalTok{rendimiento <-}\StringTok{ }\KeywordTok{factor}\NormalTok{(rendimiento,}
                      \DataTypeTok{labels =} \KeywordTok{c}\NormalTok{(}\StringTok{"bajo"}\NormalTok{, }\StringTok{"medio"}\NormalTok{, }\StringTok{"alto"}\NormalTok{),}
                      \DataTypeTok{ordered =} \OtherTok{TRUE}\NormalTok{)}

\NormalTok{rendimiento}
\end{Highlighting}
\end{Shaded}

\begin{verbatim}
##  [1] bajo  alto  bajo  medio medio bajo  medio medio alto  medio
## Levels: bajo < medio < alto
\end{verbatim}

\bibliography{book.bib,packages.bib}


\end{document}
